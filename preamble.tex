\newcommand{\AJ}{\aj}

% from http://mintaka.sdsu.edu/GF/bibliog/latex/floats.html
% Alter some LaTeX defaults for better treatment of figures:
% See p.105 of "TeX Unbound" for suggested values.
% See pp. 199-200 of Lamport's "LaTeX" book for details.
%   General parameters, for ALL pages:
\renewcommand{\topfraction}{0.9}	% max fraction of floats at top
\renewcommand{\bottomfraction}{0.8}	% max fraction of floats at bottom
%   Parameters for TEXT pages (not float pages):
\setcounter{topnumber}{2}
\setcounter{bottomnumber}{2}
\setcounter{totalnumber}{4}     % 2 may work better
\setcounter{dbltopnumber}{2}    % for 2-column pages
\renewcommand{\dbltopfraction}{0.9}	% fit big float above 2-col. text
%\renewcommand{\textfraction}{0.07}	% allow minimal text w. figs
%   Parameters for FLOAT pages (not text pages):
%\renewcommand{\floatpagefraction}{0.7}	% require fuller float pages
% N.B.: floatpagefraction MUST be less than topfraction !!
%\renewcommand{\dblfloatpagefraction}{0.7}	% require fuller float pages
% remember to use [htp] or [htpb] for placement

% \afterpage{clearpage}

\usepackage{amssymb,amsmath}
\usepackage{multirow}
\usepackage{natbib}
\usepackage{xspace}
\usepackage{dcolumn}
\usepackage{afterpage}
\usepackage{calc}
% See http://ringlord.com/latex-to-pdf-howto.html
\pdfcompresslevel=9
%\usepackage[pdftex]{graphicx}
%\usepackage{thumbpdf}
%\usepackage[pdftex]{color}
\usepackage{graphicx}
\usepackage{color}
\definecolor{urlcolor}{rgb}{0,0,0}
\definecolor{filecolor}{rgb}{0,0,0}
\definecolor{linkcolor}{rgb}{0,0,0}
\definecolor{anchorcolor}{rgb}{0,0,0}
\definecolor{citecolor}{rgb}{0,0,0}
\definecolor{menucolor}{rgb}{0,0,0}
%\definecolor{runcolor}{rgb}{0,0,0}
\usepackage[pdftex,
        colorlinks=true, % color (true) or boxes (false)
        urlcolor=urlcolor,       % \href{...}{...} external (URL)
        filecolor=filecolor,     % \href{...} local file
        linkcolor=linkcolor,       % \ref{...} and \pageref{...}
        anchorcolor=anchorcolor,
        citecolor=citecolor,
        menucolor=menucolor,
%        runcolor=runcolor,
        pdfproducer={pdfLaTeX},
        pdfpagemode=None,
        bookmarksopen=true]{hyperref}
% verbose=true

\newlength{\figunit}
\setlength{\figunit}{0.48\textwidth}

% Marginal notes
\newcommand{\note}[1]{\marginpar{{\footnotesize #1}}}

\newcommand{\unit}[1]{\ensuremath{\mathrm{#1}}}
\renewcommand{\arcmin}{\unit{arcmin}}
\renewcommand{\mag}{\unit{mag}}
\newcommand{\nanometers}{\unit{nm}}
\newcommand{\degrees}{\unit{degrees}}
\renewcommand{\deg}{\unit{degree}}
\newcommand{\RA}{\unit{RA}}
\newcommand{\Dec}{\unit{Dec}}
\newcommand{\milliseconds}{\unit{ms}}

\newcommand{\scamp}{\texttt{Scamp}\xspace}

\newcommand{\captionpart}[1]{\textbf{#1}}

\newcommand{\metadata}{meta-data\xspace}

\newcommand{\xref}[2]{\mbox{#1~\ref{#2}}\xspace}
\newcommand{\eqnref}[1]{\xref{equation}{#1}}
\newcommand{\secref}[1]{\xref{section}{#1}}
\newcommand{\subsubsecref}[1]{\secref{#1}}
\newcommand{\Fig}{Figure\xspace}
\newcommand{\fig}{figure\xspace}
\newcommand{\figs}{figures\xspace}
\newcommand{\Figs}{Figures\xspace}
\newcommand{\figref}[1]{\xref{\fig}{#1}}
\newcommand{\Figref}[1]{\xref{\Fig}{#1}}
\newcommand{\Chapref}[1]{\xref{Chapter}{#1}}
\newcommand{\chapref}[1]{\xref{chapter}{#1}}

\newcommand{\healpix}{HEALPix\xspace}
\newcommand{\healpixels}{HEALPixels\xspace}

\newcommand{\squareparens}[1]{\left[ \, #1 \, \right]}
\newcommand{\expect}[1]{\mathbb{E}\squareparens{#1}}
\newcommand{\expectover}[2]{\mathbb{E}_{#1}\squareparens{#2}}
\newcommand{\given}{\,\vert\,}

\newcommand{\aposteriori}{\textsl{a posteriori}\xspace}
\newcommand{\MAP}{maximum \aposteriori\xspace}

\newcommand{\abs}[1]{\left\lvert #1 \right\rvert}
\newcommand{\ngauss}{\mathcal{N}}
\newcommand{\pgauss}[3]{\ngauss(#1 \, | \, #2,\, #3)}
\newcommand{\ie}{\textit{i.e.}}
\newcommand{\viceversa}{vice versa}
\newcommand{\eg}{eg.}
\newcommand{\dd}{\mathrm{d}}
\newcommand{\krondelta}[2]{\delta(#1, #2)}
\newcommand{\thetamax}{\theta_{\mathrm{max}}}

% tech-report stuff
\newlength{\gridfigwidth}
\setlength{\gridfigwidth}{3.05in}

\newlength{\quadfigwidth}
\setlength{\quadfigwidth}{3.05in}

\newlength{\densityfigwidth}
\setlength{\densityfigwidth}{6in}


% tech-report, verify
\newcommand\tstrut{\rule[-0.5ex]{0pt}{3ex}}
\newcommand{\truepos}{\ensuremath{\mathrm{TP}}}
\newcommand{\falsepos}{\ensuremath{\mathrm{FP}}}
\newcommand{\trueneg}{\ensuremath{\mathrm{TN}}}
\newcommand{\falseneg}{\ensuremath{\mathrm{FN}}}

\newcommand{\starlabel}[1]{\ensuremath{#1}}
\newcommand{\starA}{\starlabel{A}}
\newcommand{\starB}{\starlabel{B}}
\newcommand{\starC}{\starlabel{C}}
\newcommand{\starD}{\starlabel{D}}
\newcommand{\xC}{\ensuremath{x_\starC}}
\newcommand{\yC}{\ensuremath{y_\starC}}
\newcommand{\xD}{\ensuremath{x_\starD}}
\newcommand{\yD}{\ensuremath{y_\starD}}



% KDTREE stuff
\newcommand{\codecomment}[1]{\# \hackbold{#1}}


% VERIFY stuff
\newcommand{\fg}{\ensuremath{F}}
\newcommand{\bg}{\ensuremath{B}}
\newcommand{\data}{\ensuremath{D}}

\newcommand{\fgone}{\ensuremath{F_1}}
\newcommand{\bgone}{\ensuremath{B_1}}
\newcommand{\fgsymm}{\ensuremath{F_s}}
\newcommand{\bgsymm}{\ensuremath{B_s}}
\newcommand{\bgadapt}{\ensuremath{B_a}}
\newcommand{\fgtwo}{\ensuremath{F_2}}

%\newcommand{\fgsymm}{\ensuremath{F\!{}_s}}
%\newcommand{\bgsymm}{\ensuremath{B\!{}_s}}
%\newcommand{\underr}{\ensuremath{U_r}}
%\newcommand{\undert}{\ensuremath{U_t}}
%\newcommand{\underfg}{\ensuremath{U}}

\newcommand{\tableheaderx}[1]{\multicolumn{1}{|c|}{\textbf{#1}}}
\newcommand{\tableheader}[1]{\multicolumn{1}{c|}{\textbf{#1}}}
\newcommand{\mtableheader}[2]{\multicolumn{#1}{|c|}{\textbf{#2}}}

%\newcommand{\an}{\emph{Astrometry.net}}
\newcommand{\an}{\textsl{Astrometry.net}\xspace}
\newcommand{\antitle}{Astrometry.net\xspace}
\newcommand{\kdtreetitle}{KD-tree\xspace}
\newcommand{\kdtreestitle}{kd-trees\xspace}
\newcommand{\kdtree}{kd-tree\xspace}
\newcommand{\kdtrees}{kd-trees\xspace}
\newcommand{\Kdtree}{Kd-tree\xspace}
\newcommand{\Kdtrees}{Kd-trees\xspace}

\DeclareMathOperator*{\argmax}{argmax}

% make the space between the number and % sign a bit smaller than ~.
\newcommand{\percent}{\!\%}

\newcommand{\floor}[1]{\ensuremath{\left\lfloor #1 \right\rfloor}}
\newcommand{\ceil}[1]{\ensuremath{\left\lceil #1 \right\rceil}}
\newcommand{\half}{\ensuremath{\frac{1}{2}}}
\newcommand{\smallhalf}{\ensuremath{\textstyle \frac{1}{2}}}
%\newcommand{\deg}{\ensuremath{^\circ}}
\newcommand{\latin}[1]{\textit{#1}}
\newcommand{\etal}{\emph{et al.}~}

\newcommand{\hackbold}[1]{\makebox[0pt][l]{\hspace{0.2pt}#1}#1}
\newcommand{\class}{\hackbold{class} }
\newcommand{\func}{\hackbold{def} }
\newcommand{\spc}{ }

\newcommand{\mindist}{\texttt{mindist}\xspace}
\newcommand{\maxdist}{\texttt{maxdist}\xspace}
\newcommand{\trick}[1]{\subsection{#1}}
\DeclareMathOperator*{\argmin}{argmin}

\newenvironment{codesize}
               {\begin{scriptsize}}
               {\end{scriptsize}}

\newcommand{\tabstops}{123\=123\=123\=123\=123\=\kill}

\newenvironment{pcode}
               {\begin{center}% \fbox{%
			     \linespacing{1.25}
               \begin{minipage}[c]{\textwidth}%
               \begin{codesize}%
                   \tt
                   \begin{tabbing}
                     \tabstops
               }
               {\end{tabbing}%
                 \end{codesize}%
                 \end{minipage}%
                 \end{center}%
               }

