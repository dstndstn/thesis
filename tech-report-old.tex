% notes
% -----
% - use \figurename not figure !
% - use \USNOB and \deg and \arcmin and \starA \starB and etc...
% - compile with make, not LaTeX

% to-do
% -----
% - DWH: various citation references don't work -- see .aux file.
% - DSTN: make figures
%   - the USNO-B has been uniformized
%   - the catalog of quads is uniform
%   - a close-up of the sky showing the quad footprints in some area
%   - some figures relating to the calibration of SDSS and GALEX.
% - Discussion needs outline
% - Discussion needs to be written
% - DWH must deal with all comments marked ``DWH''
% - DSTN must deal with all comments marked ``DSTN''

\documentclass[12pt,preprint]{aastex}
\usepackage{graphicx}
\newcommand{\An}{\textsl{Astrometry.net}}
\newcommand{\kdtree}{KD-tree}
\newcommand{\unit}[1]{\mathrm{#1}}
\newcommand{\percent}{\unit{percent}}
\renewcommand{\%}{\percent}
\renewcommand{\arcmin}{\unit{arcmin}}
\renewcommand{\deg}{\unit{deg}}
\newcommand{\etal}{\latin{et al.}}
\begin{document}

\title{\textsl{Astrometry.net}: \\
       Blind astrometric calibration of arbitrary astronomical images}
\author{Dustin~Lang\altaffilmark{1},
        David~W.~Hogg\altaffilmark{2,3},
        Keir~Mierle\altaffilmark{1,4},
        Michael~Blanton\altaffilmark{2}, \&
        Sam~Roweis\altaffilmark{1,4}
}
\altaffiltext{1}{Department of Computer Science, University of Toronto}
\altaffiltext{2}{Center for Cosmology \& Particle Physics, Department of Physics, New York University}
\altaffiltext{3}{to whom correspondence should be addressed: david.hogg@nyu.edu}
\altaffiltext{4}{Google Inc., Mountain View, CA}

\begin{abstract}
We have built a reliable and robust system that takes as input an
astronomical image, and returns as output the pointing, scale, and
orientation of that image (the astrometric calibration or WCS
information).  The system requires no first guess, and works with the
information in the image pixels alone.  The success rate is
$>99.9~\percent$ for contemporary near-ultraviolet and visual imaging
survey data, with no false positives.  We are using this system to
generate consistent and standards-compliant meta-data for all digital
and digitized astronomical imaging, no matter what its archival state,
including imaging from plate repositories, individual scientific
investigators, and amateurs.  This is the first step in a program of
making all of the world's heterogeneous astronomical data
interoperable.
\end{abstract}

\section{Introduction}

In this report, we describe a successful enterprise in the area of
pattern recognition, or object recognition in computer vision: We can
take any image of the night sky, and return its pointing, scale, and
orientation---its astrometric calibration meta-data---with a very high
success probability and essentially no false positives.  This is not
just a parlor trick: Successful recognition of an astronomical image
brings with it a great deal of meta-data, including the precise
astrometric calibration of the image, the objects contained in the
image, and the information required to estimate the photometric
bandpass of the image, its sensitivity, and the time at which it was
taken \cite{barron08b}.  These meta-data are required for reliable
scientific use of the image.

\paragraph{Our problem is easy.}
There are many obstacles to general systems that can
recognize large classes of objects in images.  In general, objects
move around relative to their backgrounds and are randomly occluded by
complex objects in the foreground, they are often articulated or
deformable (think limbs), they can be viewed from any angle, there are
large changes possible in the same object or object class (flowers
bud, bloom, and wilt), specific feature matching and identification is
non-trivial (find the nose), and they are subject to dramatic changes
in lighting.  We have none of these problems when we consider
astronomical images.

The stars are extremely distant, so our viewpoint is effectively
fixed.  Of course we are moving as are the stars in our Galaxy, so
there are significant (and extremely important) motions on human
timescales.  However, these configurational and viewpoint changes
create individual-star angular movements that are very small relative
to the mean inter-star separations in our catalogs (this is the
relevant comparison, since it determines the changes to the apparent
shapes of asterisms), and smaller than or comparable to the angular
resolution of typical imaging.  The stars create their own light,
there are no differences of illumination, only differences of relative
brightness in different observational bandpasses.  Stars appear
to modern instruments as unresolved points, much brighter than the
background level (in most cases).  This makes them very easy to detect
and measure; we don't have the usual problems of feature
identification.

These properties makes our problem much easier than typical image
interpretation tasks.  We also benefit from the fact that large,
uniform catalogs exist, containing $>10^8$ stars observed in multiple
bandpasses and over time periods measured in decades.  This means that
when we \emph{do} find a matching asterism, we can say quite a bit
about it.

\paragraph{Our problem is hard.}
On the other hand, we face a few challenges that are significant.  The
first is that the sky is big: A typical astronomical image we consider
includes substantially less than $10^{-6}$ of the sky, and contains
thousands of stars, only a fraction of which have been previously
catalogued.  Furthermore, the stars have been catalogued using data in
a bandpass which, in general, can be very different from the bandpass
in which the input image was taken.  There can be arbitrary variation
in the sensitivity to particular stars or the probability that a
bright star in the image is represented in the catalog.

For many images we see, on the web, on magnetic tapes in academic
offices, and on hard drives or CDs in the basements of amateurs, we do
not know very much: We don't know when they were taken, what telescope
was used, what bandpass they are in, how big they are (in an angular
sense), where they are centered on the sky, the exposure time, or what
mode the telescope or image rotator was in.  This leads to an enormous
heterogeneity in what we hope to see, and sets a high bar if we want
to create a complete census of the world's astronomical data.

\paragraph{The need is great.}
There is a sense in which astrometric calibration is trivial:
Observers know what they are observing, and telescopes know where they
are pointing!  However, astrometric calibration is also non-trivial:
The vast majority of astronomical images in existence have no---or
worse, wrong---astrometric meta-data.  There are several reasons for
this: The WCS standard for astrometric meta-data appeared relativey
late (Greisen \& Calabretta 2002, Calabretta \& Greisen 2002); many
astronomers use their own personal conventions for measuring and
recording the information; some projects do not require precise global
coordinates for any image so do not bother recording them; many
telescopes have control systems that drift relative to the sky as
observers tweak the pointing; and many telescope--instrument
combinations output images with meta-data that are not
standards-compliant or simply wrong.

A system to automatically and blindly calibrate all astronomical
imaging coming off of instruments at each of the world's observatories
could catch telescope drifts and faults in real time, potentially
saving significant amounts of observing time and operational costs.
Similarly, if observers are given already-calibrated data from the
telescope, then they can avoid this task and spend their time and
grant money doing more illuminating data analysis.

Amateur observers, many of whom take professional-grade data and some
of whom publish IAU circulars and discoveries, do not usually have the
tools they need to produce accurate and standards-compliant
astrometric calibrations for their images.  A system that provides
these could bring a whole treasure trove of data to light, backed by a
community that has always had enthusiasm for supporting basic
research.

Historical plate repositories are filled with photographic plates
which represent an irreproducible record of the sky.  Efforts to digitize
these collections have been hindered by the fact that it is difficult to
record in a reliable way the plates' meta-data.
This is because the meta-data exist in hand-written logs, not all of
which survive, and not all of which are legible and organized.
Furthermore, these logs contain many errors, and more errors are
introduced in data entry.  The barriers to making these meta-data
accurate and accessible obscure these data---which provide our only
record of the sky's time history---from scientific view.  Imagine
looking for precursor activity for some future Galactic supernova.  It
may have appeared in hundreds of plates over the last hundred years.
There is no way to find those data without reliable and
standards-compliant astrometric meta-data.

\paragraph{Prior work.}
An early application of blind astrometric matching was for attitude
estimation of spacecraft: by identifying constellations, the pose of
the camera can be determined \citep{liebe1993}.  In such systems,
triangles (three-star asterisms) are typically used
(e.g. \citealt{junkins1977}).  Triangles are effective in this domain
because the images span tens of $\deg$ and contain only dozens of
very bright stars: the search space is small.  Furthermore, these
systems are not fully ``blind'' in that they are designed for
particular cameras, the specifications of which are available to the
system designers.

Triangle-based approaches have also been used to solve the astrometry
problem when a good initial estimate is available \citep{palbakos}.
Because the search domain is limited to a small area around the
initial estimate, the triangle-based approach is still effective.
Completely blind systems have been attempted previously (for example,
\citealt{harvey} and references therein) but none we know of have been
able to achieve the scalability and fidelity of our approach.

Both the triangle matching approach and ours (described below) are
based on ``geometric hashing'' (for example, \citealt{lamdan1990},
\citealt{huttenlocher1990}).  Our system uses the same two-step
approach used by these systems where a set of hypotheses are generated
from sparse matches, then a second stage does detailed verification.

There are automated calibration systems that refine the astrometric
WCS of an image, given a good first guess (for example,
\citealt{valdes, mink, bertin}).  These systems are reliable and
robust, but they require a reasonable first guess about the image
pointing, orientation, and scale.  These systems are not directly
related to the system described below, because, in some system, this
system \emph{creates} that good first guess, for images that have no
first guess, or for which that first guess is untrusted.

\section{Methods}

Our basic approach, as outlined above, involves two components.  The
first is an indexing system for asterisms, which can examine a query
image and very quickly generate candidate explanations for that image,
where each explanation consists of a proposed location, orientation
and field of view on the celestial sphere.  The second component is a
verification criterion, which can very accurately decide if any
proposed explanation is a true positive match to the query image.

The system we developed begins by extracting the two-dimensional
positions of stars in a query image, typically yielding a few hundred
stars localized to pixel accuracy or better.  Then, using a subsets of
those stars (subsets of four in the experiments here), the indexing
system generates \emph{hypotheses}---proposed alignments of the query
image to the sky.  We assign a score to each hypothesis by measuring
its ability to predict the locations of other stars in the image
(taking into account that we cannot expect the image and catalog stars
to overlap exactly). The number of such overlaps, compared with the
number of image and catalog objects and the pixel tolerance for
alignment allows us to accurately estimate the chance of the
hypothesis being a false match; we output a result only when this
chance is vanishingly small.  We continue generating and testing
hypotheses until we find one with sufficient confidence and output it
as our chosen match.  In some cases, we never find a hypothesis with
sufficient confidence (or we give up searching before we find one), but
our thresholds are set conservatively enough that we almost never
produce a false positive match.

\subsection{Star detection}

The system automatically detects compact objects each input image and
centroids them to yield the pixel-space locations of stars. This
problem is an old one in astronomy; the challenge here is to perform
it robustly on the large variety of input images the system
faces. Luckily, because the rest of the system is so robust, it can
handle detection lists that are missing a few stars or have some
contaminants.

The first task is to identify (detect) localized sources of light in
the image.  Most astronomical images exhibit sensitivity variations as
well as scattering, and we need to find peaks on top of such
variations, so first we subtract off a median-smoothed version of the
image to ``flatten'' it. Next, to find statistically significant
peaks, we need to know the approximate noise level.  We find this by
choosing a few thousand random pairs of pixels separated by five rows
and columns, calculating the difference in the fluxes for each pair,
and calculating the variance of those differences, which is
approximately twice the variance $\sigma^2$ in each pixel. At this
point, we identify pixels which have values in the flattened image
that are $>8\,\sigma$, and connect detected pixels into individual
detected objects.

The second task is to find the peak or peaks in each detected object.
This determination begins by identifying pixels that contain larger
values than all of their neighbors. However, keeping all such pixels
would retain peaks that are just due to uncorrelated noise in the
image and individual peaks within a single ``object.'' To clean the
peak list, we look for peaks that are joined to smaller peaks by
saddle points within $3\,\sigma$ of the larger peak (or $1~\percent$
of the larger peak's value, whichever is greater), and trim the
smaller peaks out of our list.

Finally, given the list of all peaks, the third task is to centroid
the star position at sub-pixel accuracy. Following previous work
\citep{lupton}, we take a $3\times 3$ grid around each star's peak
pixel, effectively fitting a Gaussian model to the nine values, and
using the peak of the Gaussian. Occasionally this procedure produces a
Gaussian peak outside the $3\times 3$ grid, in which case we default
to the peak pixel, although such cases are virtually always caused by
image artifacts.  This procedure produces a set of $x$ and $y$
positions in pixel coordinates corresponding to the position of
objects in the image.

\subsection{Hashing of asterisms to generate hypotheses}

Hypotheses about the location of an astronomical image live in the
continuous 4-dimensional space of position on the celestial sphere
(pointing of the camera's optical axis), orientation (rotation of the
camera around its axis), and field of view (solid angle subtended by
the camera image).  If hypothesis space were discretized to
approximately the angular scale at which our system operates, there
would be $>10^{20}$ discrete possibilities.  Local refinement of a coarse
initial guess might reduce the number of required starting
possibilities to ``only'' a few billion, but even then, exhaustive
enumeration followed by verification of each hypothesis would be
impractical.  We need a fast search heuristic that ranks the
order in which we check hypotheses and that almost always proposes the
correct hypothesis early enough that we have the resources to discover
it.

We have solved this massive search problem using a continuous
geometric hash code. This code computes local ``shape fingerprints''
by mapping the relative positions of groups of four nearby stars
(``quads'') into 4-dimensional continuous-valued codewords (points in
a 4-dimensional vector space).  The geometric hash is, essentially,
the positions $(\xC,\yC)$ and $(\xD,\yD)$ of two of the four stars
(called ``$\starC$'' and ``$\starD$'')
in a coordinate system defined by the two
most widely separated stars (called ``$\starA$'' and ``$\starB$'').  This is
illustrated in \figurename~\ref{fig:quad}.  The code has some
symmetries; swapping $\starA$ and $\starB$
converts a code $(\xC,\yC,\xD,\yD)$ into
$(1-\xC,1-\yC,1-\xD,1-\yD)$ while swapping $\starC$ and $\starD$ converts
$(\xC,\yC,\xD,\yD)$ into $(\xD,\yD,\xC,\yC)$. In practice, we 
break this symmetry by demanding that $\xC \le \xD$ and that
$\xC + \xD \le 1$; we consider only the permutation that satisfies these
conditions (within noise tolerance).

This mapping has several properties that make it well suited to our
indexing application. First, the code vector is \emph{invariant} to
translation, rotation and scaling of the star positions so that it can
be computed using only the relative positions of the four stars in any
coordinate system (including pixel coordinates in a query image).
Second, the mapping is \emph{smooth}: small changes in the relative
positions any of the stars result in small changes to the components
of the code vector; this makes the codes resilient to small amounts of
positional noise in star positions.  Third, if stars are uniformly
distributed on the sky (at the angular scale of the quads being
indexed), codes will be uniformly distributed in (and thus make
good use of) the 4-dimensional code-space volume.

Noise in the image and distortion caused by the atmosphere and telescope optics
lead to noise in the measured positions of stars in the image.  In general this
noise causes the stars in a quad to move slightly with respect to each other,
which yields small changes in the shape fingerprint of the quad.  Therefore, we
must always match the image fingerprint with a \emph{neighbourhood} of
fingerprints in the index.

The standard geometric hashing ``recipe'' \citep{geometrichashing}
would suggest using triangles rather than quads.  However, the
positional noise level in typical astronomical images is sufficiently
high that triangles are not distinctive enough.  Since we want to
recognize a large number of images, our index must contain many
triangles; hence the neighbourhood in the index that matches any given
query triangle will contain many thousands of matching triangles, the
vast majority of which will be coincidental (false) matches.  By using
quads instead of triangles, we nearly \emph{square} the
distinctiveness of our features: a quad can be thought of as two
triangles that share a common edge.  Since the feature space is
larger, the neighbourhood around any given query quad contains many
fewer coincidentally-matching index quads.

In principle, we could also do quintuples, which would be even more
constraining.  However, there are two disadvantages to increasing the
number of stars in the indexed $k$-star asterism.  The first is that
the probability that all $k$ of the stars in an indexed asterism
appear in the image and that all $k$ of the stars in a look-up
asterism appear in the index catalog both decrease with increasing
$k$.  For images taken at wavelengths far from the catalog wavelength,
or shallow images, this consideration can become severe.  The second
disadvantage is that near-neighbor lookup, even with a \kdtree,
becomes more time-consuming with increasing dimensionality, and the
dimensionality of the index space is $2\,k-4$.

To actually build our indexing system, we place the quad-based,
geometrically invariant local hash codes into a precomputed index. In
a classic index on a set of files or documents, we would decide on a
set of indexing terms (for example, many thousands of vocabulary words
or short tuples of common words) and then sweep over the entire
corpus, making lists of the documents containing each term.  To
quickly find all the documents containing, for example, the words
``computer'', ``science'', and ``astronomy'' we do not need to search
the entire corpus; instead we invert the query, retrieving the
precomputed list of documents mentioning ``astronomy'' and
intersecting it with the lists for the terms ``computer'' and
``science''.  Somewhat confusingly, this indexing strategy is
sometimes referred to as an ``inverted index'' or an ``inverted file
system'' in the computer science literature \citep{invertedpaper}, but
it serves the same function as the index in a book.  Analogously,
our system divides the sky into millions of equal-area patches,
somewhat smaller than the angular scale of the query images we plan to
localize at run time. Within each patch we select a number of
quadruples of nearby stars from our reference catalog, and use our
geometric hash function to compute a 4-dimensional code for each such
quad.  We enter these codes into a code-space index, along with
the associated location on the sky and the identities of the four
catalog stars from which the code was created. The codes play the same
role as the index terms in a conventional document index system but
they are now continuous rather than discrete. To accommodate this, our
index data structure---a fast, memory-efficient, and pointer-free implementation
of a \kdtree \citep{lang08}---supports fast retrieval of all codes in the neighbourhood of
any 4-dimensional point, allowing us to quickly find all the places on
the sky at which a specific shape fingerprint is reproduced almost
exactly.  In effect, we \emph{do} examine the entire sky (in the same
way that building a traditional index requires looking at every
document), but we do it only once, at indexing time, to populate our
code-space and not for each query we process during run time
retrieval.

When presented with a list of stars from an image to calibrate, the
system iterates through groups of four stars, treating each group as a
quad and computing its code using the hash function applied to the
detected positions. Using the computed code, we perform a
(neighbourhood) lookup in the index, retrieving all the stored codes
that are almost identical to the computed one along with their
corresponding locations on the sky.  Each retrieved code is
effectively a \emph{hypothesis}, which proposes to identify the four
reference catalogue stars used to create the code at indexing time
with the four stars used to compute the query code. Each such
hypothesis is verified as described below.

The question of which hypotheses to check and when to check them is
a purely heuristic one. One could chose to wait until a hypothesis
has two or more ``votes'' from independent codes before checking it
or check every hypothesis as soon as it is proposed, whichever is
faster.  We prefer to order our hypotheses by trying the brightest
stars first, but this is also merely a search acceleration, which
does not impact the quality of the results.

In order for this approach to be successful, the index must balance
several properties.  These are the spatial uniformity of quads, the
brightness of the stars in quads, and the number of times any star is
used in a quad. Spatial uniformity is needed in order to calibrate
images anywhere on the sky. Since bright stars are more likely to
appear in query images, we should preferentially build quads with
them.  However, if bright stars are indexed too heavily, then a single
lost or badly-localized star in the query image leads to the loss of
many quads which could potentially correctly identify the image
location.  We achieve these objectives by first processing the
\USNOB\ \citep{usnob, barron} to select a spatially uniform subsample
of bright stars.  Then we build quads in a spatially uniform way,
starting with the bright stars, but adding the restriction that each
star can only be used in $N$ quads (typically $N=8$).

Implementation of an index that achieves these properties is made
easier if the angular scale (angular $\starA$--$\starB$ distance)
of all quads is
restricted to lie within a small range (for example, a factor of
two). Unfortunately, the very nature of the blind calibration problem
means that we do not know the angular plate scale of the query image a
priori. As a purely practical matter, we resolved this tension by
constructing several disjoint indices, each containing quads spanning
a small range of angular scales. After source extraction, each of
these indices is used independently (in parallel if possible) to
generate hypotheses for the query image. The first of these candidate
hypotheses that passes our verification test terminates the search on all
indices and is used to generate the final astrometric
calibration.  

\subsection{Verification of hypotheses}

The verification stage, which we explain now, is a straightforward
statistical test based on the chance of accidental agreement (as
measured by positional alignment of features) between any proposed
match and the query.

The statistical test we use to be certain we have a correct answer is
based on the \emph{predictive power} of a hypothesis. In general, if a
prototype selected using only a small number of features of the query
accurately predicts a large number of \emph{other} features in the
query almost exactly, it is a high-confidence match.  In our
application, this verification test is performed by proposing an
alignment of the query image to the sky using the positions of only a
small number of stars within the image---the stars in the quad that matched.
We then ask, ``if this
proposed alignment were correct, where else in the image would we
expect to find stars?''

Under the assumption that stars in the catalog are spatially uniformly
distributed at the angular scale of the query (this is enforced by us
in the regime in which our system operates), we can estimate the
probability of a false positive match under this criterion by thinking
of the $N$ detected stars in the image (excluding those used to
compute the alignment under consideration) as randomly located
``targets'' at which we randomly throw $M$ ``darts'' from the
catalog. (The number of darts $M$ is the number of catalog stars
contained in the footprint of the query image under the hypothesized
alignment.) If our positional tolerance for almost exact alignment of
a catalog star and an image star is $\theta$ (measured either in
pixels or in $\deg$) then each target has a size $\pi\theta^2$
(measured in image area or in solid angle). Assuming the entire query
image has size $\Omega$ (and the targets are small enough and few
enough for their overlap, if any, to be insignificant) then the
fractional target area is $\alpha=\pi\theta^2/\Omega$.  When either
the number of catalog stars (darts) $M$ or the number of image stars
(targets) $N$ is large compared to the number of ``hits'' $K$ (that
is, the number of image--catalog pairs aligning to within a tolerance
$\theta$), the probability is well approximated by the Poisson
distribution: $\alpha^K (MN)^K \exp(-\alpha M N)/k!$.  Thus, the
probability of \emph{accidentally} discovering an alignment that
generates $K$ hits after examining $Q$ hypothesis alignments is at
most
\begin{equation}
\frac{Q}{k!} \: \: \alpha^K \: (MN)^K \: \exp(-\alpha M N)
\end{equation}
By requiring this bound to be exceedingly small we guarantee a false
positive rate of zero for all practical purposes.

Operationally, our system computes this probability bound after
each hypothesis that it checks, and returns a verified match only if
the bound drops below some user specified false-positive rate, which
in our experiments is typically set exceedingly small (less than
$10^{-9}$).

Of course, as with all visual pattern matching problems, there is a
certain amount of spatial noise present in the precise positions of
features, both in the query and in the index prototypes. Positional
comparison must be robust to this noise at the appropriate scale.  In
astronomical images, however, the matching problem is further
complicated by the fact that for virtually any image of a certain
patch of the sky and virtually any all-sky catalog against which that
image is to be matched, there are some detected stars in the image
that are not present in the catalog (and, conversely, some stars in
the catalog that are missing from the image). This is because the
image and the catalog will have different sensitivities to stars of
different brightnesses, temperatures, and physical properties, and
both will contain a number of errors in the form of occlusions and
processing artifacts that lead to stars being incorrectly inserted and
removed.  Our verification method is robust to these sources of variability
and does not require all (or even most) known stars in the
catalog to match up with observed image stars in the query or
\latin{vice versa}.

While the verification test is crucial for maintaining the fidelity of
our system, it is our method for rapidly generating good candidate
proposals that is the key to success; it allows us to scale our system
to search the entire sky in less than a second.

For our application, a typical query might be an image about
$10~\arcmin$
on a side in which there lie few hundred catalog stars
($M$). If we work with about 100 detected stars ($N$) and require
alignment to within a few arcseconds ($\alpha=10^{-5}$), we generally
discover a match resulting in dozens of hits ($K>10$) after examining
a few tens of thousands of hypotheses ($Q<10^5$). Using the analysis
above, the chance of such a solution being a false positive is
estimated at less than $10^{-25}$ and is often closer to $10^{-100}$
when we find, say, 30 hits.

\section{Results}

\subsection{Blind astrometric calibration of the Sloan Digital Sky Survey}

We explored the potential for automatically organizing and annotating
a very large real world data set by taking a sample of
$9\times13~\arcmin^2$ $r$-band images generated by the Sloan Digital Sky
Survey \citep{sdss} and considering them as an unstructured set of
independent queries to our system.  For each SDSS image, we discarded
all meta-data, including all positional and rotational information and
the date on which the exposure was taken.  We allowed ourselves to
look only at the two-dimensional positions of detected ``stars'' (most
of which were in fact stars but some of which were galaxies or
detection errors) in the image.  Normally, our system would take
\emph{images} as input, running a series of image processing steps to
detect stars and localize their positions.  The SDSS data reduction
pipeline already includes such a process, so for this experiment we
used these detected star positions rather than processing all the raw
images ourselves.  Further experiments have shown that we would likely
have achieved similar, if not better, results by using our own image
processing software.

The index we used for this test was generated by taking a cut
of stars that are bright in the red bands from the \USNOB.
We placed a fine HEALPix \citep{healpix} grid
over the celestial sphere (with $\Nside=880$, yielding $9,292,800$
grid cells), and selected the brightest $9$ objects (using R-band
magnitudes) in each cell.  This yielded a bright, spatially uniform
set of about $85$ million stars (yielding a density of about
$0.58~\arcmin^{-2}$).
% nstars=85,510,039
 Using the same grid, we began building quads by examining the
brightest four stars in each grid cell, then the fifth brightest, and
so on, until a valid quad was built in each grid cell.  We demanded
that the quads have diameter ($\starA$-$\starB$ distance) between $4.0$ and
$5.6~\arcmin$.  We repeated this process $16$ times, shifting the
grid slightly each time, adding the constraint that no quad is built
more than once, and that no star be used in more than $8$ quads.  This
resulted in nearly 150 million quads (giving an average quad density
of $1.0~\arcmin^{-2}$).

We constructed quads from these image stars, computed their codes and
retrieved close matches to those codes from our index, checking the
corresponding hypotheses until we found one whose probability of being
a false match was less than $10^{-9}$. The results were extremely
good: of the nearly 250,000 SDSS images we queried, $99.92~\percent$ of them
(all but 186) were calibrated successfully with zero false
positives. Of the unsuccessful images, a further 80 could be
calibrated by applying our system directly to a composite image
containing data from several bands. Of the remaining 106 failures,
several show irregularities in the underlying catalogue from which our
index was built or in the query image itself.  Even the small number
of failures attributable to errors in the \USNOB\ prompted us
to investigate semi-automated methods for cleaning the catalogue by
removing spurious sources caused by diffration spikes and reflection
halos \citep{barron}.

For the particular index we built, the ``easiest'' $90~\percent$ of
the images could be successfully calibrated by looking at only the 25
brightest stars with an average amortized time of 12~ms per image.
The remaining $10~\percent$ were the ``hard'' images, which required
using up to 100 stars and took up to 150 seconds of CPU time.
Building the index can be done in a few hours on a single machine, and
calibration takes a few more hours.  We assumed that we knew the plate
scale (field-of-view) of the images to within $2~\percent$, but we
emphasize that exactly the same results would have been obtained,
albeit more slowly, if we had built a series of indexes at various
angular scales and tried to calibrate each image at all scales.  This
statement can be made because we have no false positives; if we
attempt calibration at a wide range of scales, the only successful
verification will occur at the correct scale.  Direct tests confirm
this.

\subsection{Blind astrometric calibration of Galaxy Evolution Explorer data}

% Run 132: GALEX GR2 jpegs (11076), index 500.

In order to demonstrate that our system is capable of handling imagery
outside the optical band, we applied it to images from the Galaxy
Evolution Explorer (GALEX) space telescope \citep{galex}, which
measures ultraviolet light.  We presented our system with JPEG
renderings of the $11,076$ images in GALEX Release 2 (GR2).  The only
information we gave our system about these images is that they are
between $1$ and $180~\deg$ wide (GALEX produces circular fields
with diameter about $1.1~\deg$.)

For this test, we built a set of indices that cover the range of image
scales.  Other than the number of stars and the range of quads, these
indices are built exactly the same as the ones used for the SDSS test.
In particular, they contain stars that are bright in the red band,
even though we are calibrating images whose bandpass does not overlap
the visible spectrum.  Each of these indices covers a factor of
$\sqrt{2}$ of scales ($\starA$--$\starB$ angular distances)
and was built by placing a HEALPix grid over the
sky (with different $\Nside$), taking a spatially uniform sample of
the brightest stars, then building a spatially uniform set of quads
from these stars.  The smallest-scale index contains quads with
diameter $5.6$ to $8~\arcmin$, while the largest contains quads up
to $30~\deg$.  These indices contain a total of about $85$
million stars and $150$ million quads.

% nstars=85,982,166,
% nquads=149,011,300

Of the $11,076$ images, our system correctly calibrated $11,062$ (a
success rate above $99.8~\%$), with no false positives.  Of the $14$
images that we failed to calibrate, we were able to calibrate one by
preprocessing the image (taking the red channel and downsampling by a
factor of two).  The $13$ remaining failures were all taken with a
far-UV filter; the majority of the images combine near-UV and far-UV
exposures.  We successfully calibrated $100~\percent$ of the images
with near-UV coverage.

\subsection{Blind astrometric calibration of other imagery}

We have also had excellent success using the same system for blindly
calibrating a wide class of other astronomical images including
scanned photographic plates from the Harvard Observatory archives
\citep{harvardplates}, amateur telescope shots spanning approximately
one $\deg^2$ of the sky and even photographs from consumer
digital SLR cameras, some of which span tens of $\deg$. These images
have very different exposure properties, capture light in the optical,
infrared and ultraviolet bands, and often have significant distortions
away from the pure tangent-plane projection of an ideal camera. Part
of the remarkable robustness of our algorithm, which allows it to
calibrate all such images using the same parameter settings, comes
from the fact that the hash function is scale invariant so that even
if the center of an image and the edges have a significantly different
pixel scale (solid angle per pixel), quads in both locations will
match properly into the index (although our verification criterion may
conservatively decide that a true solution with substantial distortion
is not correct). Furthermore, no individual quad or star, either in
the query or the index is essential to success. If we miss some
evidence in one part of the image or the sky we have many more chances
to find it elsewhere.

\section{Discussion}

We have described a system that performs astrometric
calibration---determination of imaging pointing, orientation, and
plate scale---blind, or without any prior information beyond the data
in the image pixels.  This system works by using indexed asterisms to
generate hypotheses, followed by quantitative verification of those
hypotheses.  The system makes it possible to vet or restore the
astrometric calibration information for astronomical images of unknown
provenance, and images for which the astrometric calibration is lost,
unknown, or untrsutworthy.

(DWH: applications in archives, amateur data)

(DWH: performance will degrade with image size, depth, quality, and
wavelength as follows.)

(DWH: this is the first step in building a complete model of the sky.)

The system described here is up on the web as a web service; point
your browser at http://astrometry.net/ to find out how to try it.
Also all of the code is available at this URL under an open-source
license.

\acknowledgments We thank Jon Barron, Doug Finkbeiner, Chris Kochanek,
Robert Lupton, Phil Marshall, Dave Monet, John Moustakas, and
Christopher Stumm for comments on and contributions to the prototype
version of the online service \An, plus our large alpha-testing team.
We thank Leslie Groer and the University of Toronto Physics Department
for use of their Bigmac computer cluster.  Lang, Mierle and Roweis
were funded in part by NSERC and CRC; Hogg was funded in part by the
National Aeronautics and Space Administration (grants NAG5-11669 and
07-ADP07-0099) and the National Science Foundation (grant
AST-0428465).  Blanton was funded in part by the NASA \textit{Spitzer
Space Telescope} (grant G03-AR-30842).

This project made use of public SDSS data.  Funding for the SDSS and
SDSS-II has been provided by the Alfred P. Sloan Foundation, the
Participating Institutions, the National Science Foundation, the
U.S. Department of Energy, the National Aeronautics and Space
Administration, the Japanese Monbukagakusho, the Max Planck Society,
and the Higher Education Funding Council for England. The SDSS Web
Site is http://www.sdss.org/.

The SDSS is managed by the Astrophysical Research Consortium for the
Participating Institutions. The Participating Institutions are the
American Museum of Natural History, Astrophysical Institute Potsdam,
University of Basel, University of Cambridge, Case Western Reserve
University, University of Chicago, Drexel University, Fermilab, the
Institute for Advanced Study, the Japan Participation Group, Johns
Hopkins University, the Joint Institute for Nuclear Astrophysics, the
Kavli Institute for Particle Astrophysics and Cosmology, the Korean
Scientist Group, the Chinese Academy of Sciences (LAMOST), Los Alamos
National Laboratory, the Max-Planck-Institute for Astronomy (MPIA),
the Max-Planck-Institute for Astrophysics (MPA), New Mexico State
University, Ohio State University, University of Pittsburgh,
University of Portsmouth, Princeton University, the United States
Naval Observatory, and the University of Washington.

This research made use of the NASA Astrophysics Data System and the
USNO Image and Catalogue Archive Service.

\begin{thebibliography}{70}
%
\bibitem[Barron \etal(2008)]{barron}
J.~T.~Barron, C.~Stumm, D.~W.~Hogg, D.~Lang, S.~Roweis,
``Cleaning the USNO-B Catalog through automatic detection of optical artifacts,''
\textit{The Astronomical Journal}
\textbf{135}, 414--422 (2008).
DOI:~10.1088/0004-6256/135/1/414
%
\bibitem[Bloom(1970)]{bloom}
B.~H.~Bloom,
``Space/time trade-offs in hash coding with allowable errors,''
\textit{Communications of the ACM}
\textbf{13:7}, 422--426 (1970).
DOI:~10.1145/362686.362692
%
\bibitem[Chazelle \etal(2004)]{bloomier}
B.~Chazelle, J.~Kilian, R.~Rubinfeld, A.~Tal,
``The Bloomier filter: an efficient data structure for static support lookup tables,''
\textit{SODA '04: Proceedings of the fifteenth annual ACM-SIAM Symposium on Discrete algorithms},
30--39 (2004).
%
\bibitem[G{\'o}rski \etal(2005)]{healpix}
K.~M.~G{\'o}rski, E.~Hivon, A.~J.~Banday, B.~D.~Wandelt, F.~K.~Hansen, M.~Reinecke, M.~Bartelmann,
``HEALPix:\ A Framework for High-Resolution Discretization and Fast Analysis of Data Distributed on the Sphere,''
\textit{The Astrophysical Journal}
\textbf{622}, 759--771 (2005).
DOI:~10.1086/427976
%
\bibitem[Harvey(2004)]{harvey}
C.~Harvey,
\textit{New Algorithms for Automated Astrometry},
MSc Thesis, Computer Science, University of Toronto (2004).
%
\bibitem[Huttenlocher \& Ullman(1990)]{huttenlocher1990}
D.~P.~Huttenlocher, S.~Ullman,
``Recognizing solid objects by alignment with an image,''
\textit{International Journal of Computer Vision}
\textbf{5:2}, 195--212 (1990).
%
\bibitem[Junkins \etal(1977)]{junkins1977}
J.~L.~Junkins, C.~C.~White, J.~D.~Turner,
``Star pattern recognition for real time attitude determination,''
\textit{Journal of the Astronautical Sciences}
\textbf{25}, 251--270 (1977).
%
\bibitem[Lamdan \etal(1990)]{lamdan1990}
Y.~Lamdan, J.~T.~Schwartz, H.~J.~Wolfson,
``Affine invariant model-based object recognition,''
\textit{IEEE Transactions on Robotics and Automation}
\textbf{6:5}, 578--589 (1990).
%
\bibitem[Lang \& Mierle(2008)]{lang08}
D.~Lang, K.~Mierle,
``Kd-trees without the Pain, but with all the Performance,''
\textit{in preparation} (2008).
%
\bibitem[Liebe(1993)]{liebe1993}
C.~C.~Liebe,
``Pattern recognition of star constellations for spacecraft applications,''
\textit{IEEE Aerospace and Electronic Systems Magazine}
\textbf{8:1}, 31--39 (1993).
%
\bibitem[Lowe(1999)]{sift}
D.~G.~Lowe,
``Object recognition from local scale-invariant features,''
\textit{International Conference on Computer Vision},
1150--1157 (1999).
%
\bibitem[Lupton \etal(2001)]{lupton}
R.~H.~Lupton, J.~E.~Gunn, Z.~{Ivezi{\'c}}, G.~R.~Knapp, S.~Kent, N.~Yasuda,
``The SDSS Imaging Pipelines,''
\textit{ASP Conf.\ Ser.\ 238:\ Astronomical Data Analysis Software and Systems X.}
(\textbf{10}) (2001).
%
\bibitem[Martin \etal(2005)]{galex}
D.~C.~Martin \etal,
``The \textsl{Galaxy Evolution Explorer:}\ A Space Ultraviolet Survey Mission,''
\textit{The Astrophysical Journal}
\textbf{619}, L1--L6 (2005). DOI:~10.1086/426387
%
\bibitem[Monet \etal(2003)]{usnob}
D.~G.~Monet \etal,
``The USNO-B Catalog,''
\textit{The Astronomical Journal}
\textbf{125}, 984--993 (2003).
DOI:~10.1086/345888
%
\bibitem[P\'al \& Bakos(2006)]{palbakos}
A.~P\'al, G.~\'A.~Bakos,
``Astrometry in Wide-Field Surveys,''
\textit{The Publications of the Astronomical Society of the Pacific}
\textbf{118}, 1474--1483 (2006).
%
\bibitem[Simcoe \etal(2006)]{harvardplates}
R.~J.~Simcoe, J.~E.~Grindlay, E.~J.~Los, A.~Doane, S.~G.~Laycock, D.~J.~Mink, G.~Champine, A.~Sliski,
``An ultrahigh-speed digitizer for the Harvard College Observatory astronomical plates,'' 
\textit{Applications of Digital Image Processing XXIX: Proceedings of SPIE}
\textbf{6312}, 17 (2006).
%
\bibitem[Wolfson \& Rigoutsos(1997)]{geometrichashing} 
H.~J.~Wolfson, I.~Rigoutsos,
``Geometric hashing: an overview,''
\textit{IEEE Computational Science and Engineering}
\textbf{4:4}, 10--21 (1997).
%
\bibitem[York \etal(2000)]{sdss}
D.~G.~York \etal,
``The Sloan Digital Sky Survey:\ Technical summary,''
\textit{The Astronomical Journal}
\textbf{120}, 1579--1587 (2000).
DOI:~10.1086/301513
%
\bibitem[Zobel \& Moffat(2006)]{invertedpaper}
J.~Zobel, A.~Moffat,
``Inverted Files for Text Search Engines,''
\textit{ACM Computing Surveys}
\textbf{38}(2), article 6 (July 2006).
%
\end{thebibliography}

\begin{figure}
\begin{center}
\includegraphics[width=3in]{figs/quadfig-crop.pdf}
\caption{Our geometric hash function converts the relative positions
  of four stars (denoted \starA, \starB, \starC, and \starD)
  into a 4-dimensional continuous
  hash-code vector. The function defines a local coordinate system
  $(x,y)$ by taking the pair of stars separated by the largest
  distance (labelled $\starA$ and $\starB$) and placing one
  of them (\starA) at $(x=0,y=0)$ and the other (\starB)
  at $(x=1,y=1)$.  Within
  that coordinate system, the positions of the remaining two stars
  (labelled $\starC$ and $\starD$)
  form a 4-dimensional code $(\xC,\yC,\xD,\yD)$ which
  encodes the relative geometric arrangement of the quadruple of
  stars, or ``quad''. By construction, this code is invariant to
  translation, rotation and scaling of the stars in the quad.
  \label{fig:quad}}
\end{center}
\end{figure}
\vspace*{-.5in}


\begin{figure}
\begin{center}
\includegraphics[width=10cm]{figs/indexing.pdf}
\caption{Our index of the sky begins with the USNO-B1.0 and Tycho-2
  catalogs (A).  This figure shows the number density of stars in
  these catalogs.  We select a bright, spatially-uniform cut of stars
  by laying down a fine HEALPix grid and selecting the brightest $N$
  stars in each grid cell (B).  To keep the file sizes manageable, we
  split the index into twelve pieces corresponding to the base-level
  HEALPixels.  There is a small margin added to each index, resulting
  in the narrow seams in the figure which mark the HEALPixel
  boundaries.  We build bright, spatially uniform, robust quads from
  these stars by again laying down a fine HEALPix grid and attempting
  to build a quad whose center of mass lies within each grid cell,
  starting with the brightest stars, but subject to the constraint
  that any given star can only be used in a small number of quads
  (hence, losing one star causes us to lose only a small number of
  quads) (C). \label{fig:indexing}}
\end{center}
\end{figure}
\vspace*{-.5in}

\begin{figure}
\begin{center}
\includegraphics[width=4.75in]{figs/fig4.png} 
\caption{System performance and parameters can be measured by
 aggregating information across a large number of solved images (a
 quarter million fields from the Sloan Digital Sky Survey for this
 figure).  Since our algorithm continues to search for a quad match
 until it succeeds, some fields solve after we have examined only a
 few stars while others require that we look at many stars before
 correctly identifying them; the vast majority of images in this
 sample solved after looking at 30 stars or less (A). In order to
 optimize our use of computational resources, we solved the images in
 two phases. The first phase required a detected hash code in the
 image to match codes in the index quite closely (left vertical line
 in B); this greatly reduces the number of matches we consider. Almost
 all fields solved in this phase.  The second phase relaxed the
 matching requirement substantially (right vertical line in B) which
 solved almost all of the remaining fields, but we checked many more
 potential matches per image.  Correspondingly, the average amount of
 computation required per image is dramatically smaller for solving
 the easiest $90~\%$ of fields than for the hardest $10~\%$ (C).  Once
 we have found a candidate match, it must pass a statistical
 verification test which, roughly speaking, computes the odds of the
 match occurring at random. We set the cutoff for this test to one in
 $10^9$ (a log odds of 9) but virtually all the fields solved with
 odds of $10^{40}$ or greater (D).  \label{fig:stats}}
\end{center}
\end{figure} 

\end{document}
